\documentclass[a4paper]{article}



%% Language and font encodings
\usepackage[english]{babel}
\usepackage[utf8]{inputenc}
\usepackage[T1]{fontenc}

%% Sets page size and margins
\usepackage[a4paper,top=2.5cm,bottom=2.5cm,left=2.5cm,right=2.5cm,marginparwidth=1.75cm]{geometry}

%% Useful packages
\usepackage{paralist}
\usepackage{amsmath}
\usepackage{bm,bbm}
\usepackage{amsthm}
\usepackage{graphicx}
\usepackage[colorinlistoftodos]{todonotes}
\usepackage[colorlinks=true, allcolors=blue]{hyperref}
\usepackage{nameref,cleveref}

\usepackage{MarkMathCmds}
\usepackage{mathtools}
\newcommand{\mat}[1]{{\mathrm{{#1}}}} % matrix

\usepackage{MarkBiblatexCmds}
\addbibresource{ref.bib}

\usepackage{enumitem}
\usepackage{epsdice}

\newcommand{\bo}{\omega}
%\newcommand{\KL}{\text{KL}}
\newcommand{\train}{\text{train}}
\newcommand{\D}{\mathcal{D}}
\newcommand{\softmax}{\text{Softmax}}

\newcommand{\logsumexp}{\text{log-sum-exp}}

\newcommand{\vfe}{\mathcal{F}_{\text{VI}}}
\newcommand{\snr}{\text{SNR}}

\newcommand{\R}{\mathbb{R}}
\newcommand{\N}{\mathcal{N}}
\newcommand{\cL}{\mathcal{L}}
\newcommand{\cO}{\mathcal{O}}
\newcommand{\svert}{~|~}
\newcommand{\td}{\text{d}}
\newcommand{\f}{\mathbf{f}}
\newcommand{\x}{\mathbf{x}}
\newcommand{\Bb}{\mathbf{b}}
%\newcommand{\sBb}{\mathtt{b}}
\newcommand{\sBb}{\mathtt{z}}
\newcommand{\bx}{\overline{\x}}
\newcommand{\bb}{\overline{b}}
\newcommand{\y}{\mathbf{y}}
\newcommand{\z}{\mathbf{z}}
\newcommand{\bu}{\mathbf{u}}
\newcommand{\bv}{\mathbf{v}}
\newcommand{\bV}{\mathbf{V}}
\newcommand{\bU}{\mathbf{U}}
\newcommand{\bk}{\mathbf{k}}
\newcommand{\w}{\mathbf{w}}
\newcommand{\W}{\mathbf{W}}
\newcommand{\ba}{\mathbf{a}}
\newcommand{\m}{\mathbf{m}}
\newcommand{\ls}{\mathbf{l}}
\newcommand{\bL}{\mathbf{L}}
\newcommand{\A}{\mathbf{A}}
\newcommand{\X}{\mathbf{X}}
\newcommand{\Z}{\mathbf{Z}}
\newcommand{\BS}{\mathbf{S}}
%\newcommand{\BA}{\mathbf{A}}
\newcommand{\BP}{\mathbf{P}}
\newcommand{\BQ}{\mathbf{Q}}
\newcommand{\Y}{\mathbf{Y}}
\newcommand{\F}{\mathbf{F}}
%\newcommand{\I}{\mathbf{I}}
\newcommand{\M}{\mathbf{M}}
\newcommand{\bp}{\overline{\p}}
\newcommand{\bz}{\mathbf{0}}
\newcommand{\bepsilon}{\text{\boldmath$\epsilon$}}
\newcommand{\bgamma}{\text{\boldmath$\gamma$}}
\newcommand{\s}{\mathbf{s}}
\newcommand{\Unif}{\text{Unif}}
\newcommand{\boh}{\widehat{\text{\boldmath$\omega$}}}
\newcommand{\bsigma}{\text{\boldmath$\sigma$}}
\newcommand{\bSigma}{\text{\boldmath$\Sigma$}}
\newcommand{\bmu}{\text{\boldmath$\mu$}}
\newcommand{\bphi}{\text{\boldmath$\phi$}}
\newcommand{\Kh}{\widehat{\mathbf{K}}}
\newcommand{\tr}{\text{tr}}
\newcommand{\tdet}{\text{det}}
% \newcommand{\KL}{\text{KL}}
\newcommand{\ind}{\mathds{1}}
\newcommand{\bc}{\mathbf{c}}
\newcommand{\reg}{\eta}
\newcommand{\weightdecay}{\lambda}
\newcommand{\h}{\mathbf{h}}

% variables
\newcommand{\mparam}{\bm{\theta}}	% model param
\newcommand{\vparam}{\bm{\phi}}	% variational param

% gradient approximation part
\newcommand{\hparam}{\bm{\varphi}}
\newcommand{\Xb}{\mathbb{X}}
\newcommand{\hgrad}{\overline{\nabla_{\x} \h}}
\newcommand{\Hmatrix}{\mathbf{H}}
\newcommand{\Grad}{\mathbf{G}}
\newcommand{\g}{\bm{g}}
\newcommand{\noise}{\bm{\epsilon}}


\newcommand{\mx}{\vm_\vx}
\newcommand{\my}{\vm_\vy}
\newcommand{\covmat}{\boldsymbol{\Sigma}}
\newcommand{\covx}{\boldsymbol{\Sigma}_{\vx\vx}}
\newcommand{\covy}{\boldsymbol{\Sigma}_{\vy\vy}}
\newcommand{\covxy}{\boldsymbol{\Sigma}_{\vx\vy}}
\newcommand{\covyx}{\boldsymbol{\Sigma}_{\vy\vx}}

\newcommand{\K}{\mathbf{K}}

\newcommand{\questionref}[1]{\Cref{#1} -- \nameref{#1}}

\newcommand{\lb}{\mathcal{L}}
\newcommand{\sumn}{\sum_{n=1}^N}

\theoremstyle{definition}
\newtheorem{question}{Question}

\newcommand{\courseprobstats}{\texttt{50008} \textit{Probability \& Statistics}}


\title{70015 Mathematics for Machine Learning: Exercises}
\author{Mark van der Wilk, Yingzhen Li\footnote{Many thanks to teaching assistants Carles Balsells Rodas, and Alex Spies for their solutions and improvements to the document.} \\ \texttt{\{m.vdwilk,yingzhen.li\}@imperial.ac.uk}}



\begin{document}
\maketitle
\tableofcontents



% \section{Background material}\todo{Adjust.}
% You will be expected to have a \emph{firm} understanding of Mathematics for Machine Learning. In the explanations, I will be manipulating probabilities and expectations freely, as discussed in Mathematics for Machine Learning. If steps are difficult, I encourage you to raise this on the course EdStem page, or during a Q\&A session.

% \begin{itemize}
% \item Basic probability: sample spaces, disjoint events (summation of probabilities), independent events (multiplication of probabilities). See \citet{walpole2012probability}, Ch2 (Imperial Library, or search Google for reading options).
% \item Probability densities. See \citet{mml} \S 6.2.
% \item Sum, product \& Bayes' rules. See \citet{mml} \S 6.3.
% \item Unconstrained continuous optimisation. See \citet{mml} \S 7.1.
% \item Linear algebra and matrix decompositions. See \citet{mml} ch 4 (and Chs 2 and 3 for basics).
% \item A familiarity with linear basis-function regression. See \citet{mml} ch 9.
% \end{itemize}


\section{Notation}
\subsection{Sets}
Throughout this course, we will be using some standard mathematical notation which may be unfamiliar to some. It's ultimately not that special or even crucial to the overall argument, but it is compact (which is practical), and it helps somewhat with practising with expressing things mathematically. Wikipedia has good definitions on these things too.
\begin{itemize}
\item Notation referring to sets of numbers, e.g.~the natural numbers $\mathbb N = \{0, 1, 2, \dots\}$, integers $\mathbb Z = \{\dots, -2, -1, 0, 1, 2, \dots\}$, or real numbers $\mathbb R$.
\item Vectors are sets containing $n$ of some type of object, like reals. We denote the set of all such sets using a superscript notation. For example, all $n$-dimensional vectors becomes $\mathbb R^n$.
\item With $x \in \mathcal S$ we denote that $x$ is an element of the set $\mathcal S$. This allows us to specify that a variable comes from a particular set (or, has a particular type), e.g.~$x \in \Reals^D$.
\item We sometimes use ``set builder'' notation. We did this informally above when defining $\mathbb N$! Usually this works by specifying elements with some property, e.g.~$\mathbf S = \{2n \,|\, n \in \mathbb N\}$, which means ``all the elements 2n such that $n$ is a natural number''. This creates the set of all even positive whole numbers.
\item We denote the union of two sets (the set with all elements that are in either set or both) as $A \cup B$. With set-builder notation this is $A \cup B = \{x\,|\,x\in A \vee x\in B\}$, where $\vee$ means ``or''.
\item We denote the intersection of two sets (the set of all elements that are in both stes) as $A \cap B = \{x\,|\,x\in A \wedge x\in B\}$.
\item For intervals of real numbers, we use brackets, $[,]$, to denote the elements in the set which are "greater than or equal to" and "less than or equal to" an element, respectively. We use parentheses, $(,)$ to denote a strict lower bound or upper bound on the set, respectively. E.g. $[1,5)$ is equivalent to $1 \leq x < 5, x\in\mathbb{R}$.
\item We use the symbol $\neg$ to denote the complement of a set. Given a set containing all elements under consideration $\Omega$, $\neg A$ contains all elements of $\Omega$ that are not in $A$, i.e.~$\neg A = \{x \in \Omega | x \notin A\}$. We can also denote this as $\neg A = \Omega\backslash A$.
\end{itemize}

\subsection{Probabilities}
In this course we will use the notation for probabilities that is common in machine learning. The main advantage is that this notation is shorter, although it does leave certain things implicit. We include this to reduce confusion.

Consider a probability space $(\Omega, \mathcal E, \mathbb P)$ with sample space $\Omega$ (all possible outcomes of a random procedure), event space $\mathcal E$ (the set of all sets of outcomes that we assign a probability to), and probability function $\mathbb P : \mathcal E \to [0, 1]$ (a function that assigns a probability to an event), with a random variable $X: \Omega \to \Reals^D$.
\begin{itemize}
\item With $\prob{E}$ we denote the probability of an event $E \in \mathcal E$, where $E$ is a set of outcomes.
\item Following the usual convention, we use the same notation when considering random variables, e.g.~$\prob{X < 2}$ is short for $\prob{\{s \in \Omega : X(s) < 2\}}$ (see \S6.1 in \courseprobstats).
\item We usually work directly with random variables, and specify all properties using a probability mass function (pmf) or probability density function (pdf). For a specific outcome of the random variable $\alpha$, we write:
\begin{align}
    &\prob{X=\alpha} = p_X(\alpha) && \text{for a pmf } p_X(\cdot) \,, \\
    &\prob{X \in [a, b]} = \int_a^b p_X(\alpha) d\alpha && \text{for a pdf $p_X(\cdot)$ with $\alpha \in \Reals$} \,, \\
    &\prob{X \in A} = \int_A p_X(\alpha) d\alpha && \text{for a pdf $p_X(\cdot)$ with $\alpha \in \Reals^D$}
    \,.
\end{align}
\item Sometimes we may write vectors in boldface, i.e.~$\mathbf x \in \Reals^D$. We won't always though, so keep track of how we define variables!
\item We generally denote outcomes of random variables without referring explicitly to the random variable itself. For example, when we refer to an outcome $\vx$, we implicitly know there is a random variable that can take this value. We usually denote this as the capital, for example here $X$.
\item Sometimes we abuse notation, and drop the random variable when denoting distributions when the argument of the function identifies it, e.g.~$p(\vx) = p_X(\vx)$.
\item If we want to be explicit about the random variable that we are evaluating the density/mass of, I will write e.g.~$p_{X,Y}(\vx,\vy) = p_{X|Y}(\vx|\vy)p_Y(\vy)$.
\item Expectations can be denoted in two ways:
\begin{align}
    &\Exp{X}{f(X)} && \text{to emphasise that X is random, if it is clear what its distribution is}\,, \\
    &\Exp{p(\vx)}{f(\vx)} && \text{to emphasise that we will be integrating over the distribution $p(\vx)$} \,.
\end{align}
In both cases this corresponds to the integral $\int p(\vx) f(\vx) \calcd\vx$.
\item Often, densities and pmfs can be discussed in exactly the same way, if we think of the density of a discrete RV as a sum of delta functions. I.e.~$p(\vx) = \sum_{o} \delta(\vx - \vx_o) p_o$, where $\{\vx_o\}$ is the set of discrete possible outcomes that $X$ can take, and $p_o$ are their corresponding probabilities. This allows us to write an expectation as an integral, regardless of whether the RV is continuous or discrete, because for discrete RVs we get:
\begin{align}
\Exp{p(\vx)}{f(\vx)} = \int p(\vx) f(\vx) \calcd\vx = \int \sum_o \delta(\vx - \vx_o) p_o f(\vx) \calcd\vx = \sum_o f(\vx_o)p_o \,.
\end{align}
(A delta function has the property that $\int_A \delta(\vx) \calcd\vx$ is 1 if $0 \in A$, and 0 otherwise. Linearity of integrals still holds. It can often be seen as the limit of a Gaussian distribution with zero variance.)
\end{itemize}


\section{Formula Sheet}
\begin{itemize}
\item Gaussian probability density function (pdf) with input $\vx \in \Reals^D$, denoted as  $\NormDist{\vx; \vmu, \covmat}$ is
\begin{align}
  p(\vx) = \NormDist{\vx; \vmu, \covmat} = (2\pi)^{-\frac{D}{2}}\detbar{\mathbf{\Sigma}}^{-\frac{1}{2}}\exp\left(-\frac{1}{2}(\vx - \vmu)\transpose\mathbf{\Sigma}^{-1}(\vx-\vmu)\right) \,.
\end{align}
\item For a joint Gaussian density
\begin{align}
\p{\begin{bmatrix}\vx \\ \vy\end{bmatrix}} = \NormDist{\begin{bmatrix}\vx \\ \vy\end{bmatrix}; \begin{bmatrix}\mx \\ \my\end{bmatrix}, \begin{bmatrix}\covx & \covxy \\ \covyx & \covy\end{bmatrix}} \label{eq:gauss-cond-joint} \,,
\end{align}
we have the conditional density
\begin{align}
\p{\vx\given\vy} = \NormDist{\vx; \quad\mx + \covxy\covy\inv(\vy-\my), \quad\covx - \covxy\covy\inv\covyx} \label{eq:gauss-cond} \,.
\end{align}
\end{itemize}





\section{Warm-up Exercises}
To start, here are some exercises which test knowledge which is assumed in the course.

\subsection{Probability Theory}
We assume that you are familiar with probability theory up to the Computing 2nd year \courseprobstats{} course. Here are some questions to serve as a refresher. Students who are not familiar with this background should refer to the notes of \courseprobstats{} or relevant chapters of \citep{mml}. \textbf{We recommend you look at these questions when/before the course starts}. If you need a refresher, or if you do not know the notation, refer to the \courseprobstats{} notes, or discuss with a TA.

\begin{question}[Set Theory and Probability]
\label{q:setsprob}
Using the three axioms of probability show that
\begin{enumerate}[label=\alph*.]
    \item Write down the sample space of a dice. In your notation, use the set A to denote the event of a 3 or 4 occurring. What is the complement of $A$, denoted $\neg A$?
    \item For a problem about lengths, we have a sample space $\Omega = [0, 1]$. For $A = (0.3, 0.4]$, what is $\neg A$?
    \item $\prob{\neg  A} = 1 - \prob{A}$
    \item $\prob{\varnothing} = 0$, where $\varnothing$ is the empty set
    \item $0 \leq \prob{A} \leq 1$
    \item $A \subseteq B \implies \prob{A} \leq \prob{B}$
    
    \textit{Hint:} Consider the following definition. $B\backslash  A = \{x\in B: x\notin A\}$
    \item $\prob{A\cup B} = \prob{A} + \prob{B} - \prob{A\cap B}$
    \item (\textbf{*}) if $\{A_i\}_{i=1}^\infty \subseteq \Omega \text{ and } A_i \subseteq A_{i+1} \forall i$ then:
\[
P\left(\bigcup_{i=1}^{\infty} A_{i}\right) = \lim_{i\xrightarrow{}\infty} \prob{A_i}
\]
\textit{Hint:} Use axiom 3. \textbf{*}: The emphasis of this course isn't on these kinds of details, even though this should be doable with 1st-year calculus.
\item For two mutually exclusive events $A, B$, what is $\prob{A \cup B}$?
\end{enumerate}
See \citet[\S6.1.2]{mml} for a general overview, and \S4, \S\S5.1-5.4 of \courseprobstats{} for more details.
\end{question}

\begin{question}[Independent events] \textbf{Independent events don't come up as much as independent random variables, so it's ok to just follow this answer, rather than spending lots of time on it.}
When tossing two coins (where we care about the order), we have a sample space $\Omega = \{HH, HT, TH, TT\}$.
\begin{enumerate}[label=\alph*.]
    \item What outcomes are contained in the event that corresponds to the the first coin being heads? We denote the event $E_{1H}$, and others similarly.
    \item If you assume that all outcomes have equal probability, show that $E_{1H}$ and $E_{2T}$ are independent.
    \item If you assume that $E_{1H}$ and $E_{2H}$ are independent and 0.5 each, show that all outomes must have equal probability.
\end{enumerate}
See \S5.3.3 in \courseprobstats{}.
\end{question}

\begin{question}[Random Variables]
\label{q:rv}
Consider throwing two fair dice.
\begin{enumerate}[label=\alph*.]
    \item What is the sample space for all outcomes that you can get from throwing two dice? We specify the probability of each outcome to be the same.
    \item Define two random variables $A,B$ which map the outcome to the face value on each die respectively. Find the probability mass function for $A$ from the probability on outcomes. The answer will work from the definition of a random variable, but you will probably intuitively get the right answer as well.
    \item Show that $A$ and $B$ are independent.
    \item Define the random variable $C = A + B$. Derive the probability mass function of $C$.
\end{enumerate}
See \S6 of \courseprobstats{}.
\end{question}


\begin{question}[Continuous Random Variables]
Consider the random variable $X$ with a probability density $p(x) = C\cdot x$ when $x \in [0, 1]$ and $0$ elsewhere.
\begin{enumerate}[label=\alph*.]
    \item Calculate $C$.
    \item Calculate $\prob{0.3 \leq X \leq 0.75}$.
    \item Calculate $\prob{X \in [0.3, 0.75] \cup [0.8, 0.9]}$.
    \item Calculate $\Exp{X}{X}$, $\Exp{X}{X^2}$, $\Var{X}{X}$.
\end{enumerate}
Check your answers by performing numerical integration, e.g.~in Python.

See \S6.3, \S7 of \courseprobstats{} or \citet[\S 6.2.2]{mml}.
\end{question}


\begin{question}[Joint Discrete Random Variables]
Consider two random variables $A,C$, where $A$ is the outcome of one die, and $C$ gives the sum of $A$ and the sum of another die $B$.
\begin{enumerate}[label=\alph*.]
    \item From intuition, write a table of $\prob{C=c|A=a}$, which we use to denote the probability of $C$ taking the value $c$, if we know that $A$ has taken the value $a$.
    \item Write a table of $\prob{C=c, A=a}$. To help you think it through, consider a tree of outcomes that can occur. This helps illustrate independence between outcomes, which helps you figure out when you can multiply probabilities.
    \item From the values in the table $\prob{C=c, A=a}$ find $\prob{2 \leq C \leq 4}$ and $\prob{2 \leq C \leq 4, 2 \leq A \leq 4}$.
\end{enumerate}
We will cover conditional probability more later, but for now just think it through.
\end{question}

\begin{question}[Multivariate Integration]
Consider two continuous random variables $X,Y$ with joint density $p(x, y) = C\cdot (x^2 + xy)$.
\begin{enumerate}[label=\alph*.]
    \item Find $C$.
    \item Find $\prob{0.3 \leq X \leq 0.5}$.
    \item Find $\prob{X < Y}$. Perform the integration twice in both orders, once integrating over x first, once by integrating over y first.
    \item \textbf{Bonus:} Convince yourself that you know how to do this for $p(x, y, z) = C\cdot (x^2 + xyz)$ as well.
\end{enumerate}
Check your answers by performing numerical integration, e.g.~in Python.
\end{question}

\begin{question}[Statistics Terminology] Recall the following statistical terminology.
\label{q:stats-term}
\begin{enumerate}[label=\alph*.]
    \item What is a statistic?
    \item What is an estimator?
    \item What is a consistent estimator?
    \item What is a sample?
\end{enumerate}
\end{question}

%%%%%%%%%%% Yingzhen's Linear Algebra warm-up questions %%%%%%%%%%

\subsection{Linear Algebra}

\begin{question}[Dot product]
\label{q:dot_product}
Compute $\x^\top \y$ where $\x = (1, -2, 5, -1)^\top$ and $\y = (0, 4, -3, 7)^\top$.
\end{question}

\begin{question}[Matrix product]
\label{q:matrix_product}
Compute $\y = A\x$ as well as the $\ell_2$ norm of $\x$ and $\y$, where
\begin{equation*}
A = \begin{pmatrix}
-1 & 4 & 7 & 2 \\
3 & -2 & -1 & 0 \\
5 & 3 & 0 & -1
\end{pmatrix}, \quad \x = (-3, 2, 1, 3)^\top.
\end{equation*}
\end{question}

\begin{question}[Basis]
\label{q:basis}
Which of the following set of vectors are basis for $\mathbb{R}^2$?
\begin{enumerate}
    \item $\{(1, 1), (1, 0) \}$
    \item $\{(2, 4), (3, -1) \}$
    \item $\{(1, -1), (0, 2), (2, 1) \}$
    \item $\{(2, -1), (-2, 1) \}$
    \item $\{(0, 3) \}$
\end{enumerate}
\end{question}

\begin{question}[Span of vectors] 
\label{q:span}
Which of the following points are within the span of $\{(-1, 0, 2), (3, 1, 0) \}$?
\begin{enumerate}
    \item $(0, 1, 1)$
    \item $(1, 1, 4)$
    \item $(2, 1, 1)$
    \item $(-3, 4, 2)$
    \item $(0, 0, 0)$
\end{enumerate}
\end{question}

\begin{question}[Rotation matrix in $\mathbb{R}^2$]
\label{q:rotation_matrix}
What is the $2 \times 2$ matrix that rotates all the non-zero vectors in $\mathbb{R}^2$ by $45^{\circ}$ counter-clockwise?

\end{question}

\begin{question}[Linear equations]
\label{q:linear_equations}
Given the following system of linear equations:
\begin{equation*}
\begin{aligned}
    x + 2y &= 2 \\
    3x + 2y + 4z &= 5  \\
    -2x + y - 2z &= -1
\end{aligned}
\end{equation*}
Answer the following questions:
\begin{itemize}
    \item[a] Writing this system in a matrix form $A\x = \mathbf{b}$ with $\x = (x, y, z)^\top$. What are $A$ and $\mathbf{b}$?
    \item[b] Solve this system, or show that the solution does not exist.
    \item[c] What is the rank of $A$?
\end{itemize}
\end{question}

\begin{question}[Eigen decomposition]
\label{q:eigen_decomp}
Consider a matrix $A \in \mathbb{R}^{d \times d}$ and assume it has an eigen decomposition of $A = Q \Lambda Q^{-1}$ where $\Lambda = \text{diag}(\lambda_1, ..., \lambda_d)$. When $A$ is symmetric we also have $Q^{-1} = Q^\top$. Answer the following questions:
\begin{itemize}
    \item[a.] If $A$ is symmetric, show that $\x^\top A \x \geq 0$ for any $\x \in \mathbb{R}^{d \times 1}$ if and only if $\lambda_i \geq 0$ for all $i = 1,..., d$.
    \item[b.] Show that $Tr(A) = \sum_{i=1}^d \lambda_i$ where $Tr(A)$ is the trace of $A$.
    \item[c.] Show that $det(A) = \prod_{i=1}^d \lambda_i$ where $det(A)$ is the determinant of $A$.
    \item[d.] Why an entry $\lambda_i$ in the diagonal matrix $\Lambda$ is one of the solutions for the equation $A\bm{q} = \lambda \bm{q}$, $\bm{q} \neq \bm{0}$?
\end{itemize}
\end{question}



\section{Lecture 1: Vectors \& Probability}
\begin{question}[Vector notation]
We define the probability density on the vector $\vx \in \Reals^3$ with all elements $0 \leq x_k \leq 1$ as
\begin{align}
p(\vx) = \frac{1}{C} (x_1^2 + x_1x_2 + x_2^2 + 2x_2x_3) \,.
\end{align}
Put this into notation that only uses $\vx$ as a single whole vector.
\end{question}

\begin{question}[Vector independence]
Consider the density on $\vx \in \Reals^4$ with all elements $0 \leq x_k \leq 1$ as
\begin{align}
p(\vx) = \vx\transpose \begin{bmatrix} 0 & 0 & 0 & 0 \\ 1 & 0 & 1 & 0 \\ 0 & 0 & 0 & 0 \\ 1 & 0 & 1 & 0\end{bmatrix} \vx \,.
\end{align}
\begin{enumerate}[label=\alph*.]
\item Rewrite the density in terms of $\tilde\vx = [x_1, x_3, x_2, x_4]\transpose$. Note that you can do this by a substitution $\vx = \mat P \tilde\vx$, where $\mat P$ is a permutation matrix. You will see that you just need to swap the relevant rows and columns of the matrix. However, make sure that you understand the mathematical steps that really show this.
\item Divide up $\vx$ into two sub vectors $\vy = [x_2, x_4]\transpose$ and $\vz = [x_1, x_3]\transpose$. Show that $\vy \ci \vz$, i.e.~that they are independent.
\end{enumerate}
\end{question}

\begin{question}[Noise conditional independence]
Consider the probability of the data in linear regression, for a fixed setting of the parameters $\vtheta$ and given inputs $\mat X \in \Reals^{N\times D}$ where $\mat X = \{\vx_1, \dots, \vx_N\}$:
\begin{align}
p(\vy|\vtheta,\mat X) = \NormDist{\vy; \vx, \sigma^2 \eye}
\end{align}
Show that all $y_n$s are independent, for a fixed setting of the parameters $\vtheta$ and given inputs $\mat X$.
\end{question}

\begin{question}[Maximum likelihood revision] For a Gaussian distribution with mean $\mu$ and variance $\sigma^2$.
\begin{enumerate}[label=\alph*.]
\item Derive the probability distribution for $N$ iid draws.
\item Derive the maximum likelihood estimator for the mean $\mu$.
\end{enumerate}
\end{question}

\begin{question}[Maximum likelihood and minimum loss]
Show that the solution to the Maximum Likelihood estimator for linear regression is the same as the minimum squared loss estimator.
\end{question}

% \begin{question}[]
% Consider sampling a bent coin twice, with a probability of heads $p_H = 0.9$.
% \begin{enumerate}[label=\alph*.]
% \item 
% \end{enumerate}
% \end{question}








%%%%%%%%%%%%%%%%%%%%%%%%%%%%%%%%%%%



% \section{Lecture 1: Multivariate Probability}
% \begin{question}[Notation]
% \end{question}




\section{Warm-up Exercises Answers}
\subsection{Probability Theory}
\paragraph{\questionref{q:setsprob}}
\begin{enumerate}[label=\alph*.]
    \item We can choose any representation denoting the events, e.g. using abstract symbols $\Omega=\{ \epsdice{1}, \epsdice{2}, \epsdice{3}, \epsdice{4}, \epsdice{5}, \epsdice{6} \}$. Alternatively, we can represent each of the outcomes as a number $\Omega = \{1, 2, 3, 4, 5, 6\}$.
    
    Following the latter notation, $A=\{3, 4\}$, and $A=\{1, 2, 5, 6\}$.

	    
    
	\item Length problem with sample space $\Omega=[0,1]$.
	
	$\neg A = [0, 0.3] \cup (0.4,1]$
    
    \item $P(\neg  A) = 1 - P(A)$
    
    Since $\neg A$ and $A$ are mutually exclusive: $A \cup \neg A = \Omega$ and $A \cap \neg A = \varnothing$.

By combining axiom 2 and 3: $P(A) + P(\neg A) = P(A \cup \neg A) = P(\Omega) = 1$

Thus: $P(\neg A) = 1 - P(A)$ 

    \item $P(\varnothing) = 0$, where $\varnothing$ is the empty set
    
Given the sample space, $\Omega$, its complementary is the empty set $\varnothing$. 

We use property (c) and axiom 2: $ P(\varnothing) = 1 - P(\Omega) = 1 - 1 = 0$.

    \item $0 \leq P(A) \leq 1$
    
We use property (c) and axiom 1. 

Consider an event $A$, where $P(A) \geq 0$ and $P(\neg A) \geq 0$ by axiom 1. 

Then, $P(\neg A) = 1 - P(A) \geq 0 \implies 1 \geq P(A)$.

By joining both inequalities, $0 \leq P(A) \leq 1$.

    \item $A \subseteq B \implies P(A) \leq P(B)$
    
    \textit{Hint:} Consider the following definition. $B\backslash  A = \{x\in B: x\notin A\}$

Assume $A \subseteq B$ and construct $B$ as the union of two disjoint sets: $B = B\backslash A \cup A$.

Then, $B\backslash A \cap A = \varnothing$ by definition of $B\backslash A$. By axiom 1, we have $P(B\backslash A) \geq 0$. 

Use axiom 3: $P(B) = P(B\backslash A) + P(A) \geq P(A) \implies P(A) \leq P(B)$.


    \item $P(A\cup B) = P(A) + P(B) - P(A\cap B)$.

Define the union $(A\cup B)$ in terms of two disjoint sets. $(A \cup B) = A \cup B\backslash A$, where $A \cap B\backslash A = \varnothing$. 

Use axiom 3: $P(A \cup B) = P(A) + P(B\backslash A)$.

To compute $P(B\backslash A)$, we define B in terms of A, and the union of two disjoint sets: $B = (B \cap A) \cup (B\backslash A)$, where $(B \cap A) \cap (B\backslash A) = \varnothing$ by definition. 

Use axiom 3 again: $P(B) = P(B \cap A) + P(B\backslash A) \implies P(B\backslash A) = P(B) - P(B \cap A)$.

Finally: $P(A \cup B) = P(A) + P(B\backslash A) = P(A) + P(B) - P(B \cap A)$.

    \item (\textbf{*}) if $\{A_i\}_{i=1}^\infty \subseteq \Omega \text{ and } A_{i-1} \subseteq A_{i}\quad \forall i>0$ then:
\[
P\left(\bigcup_{i=1}^{\infty} A_{i}\right) = \lim_{i\xrightarrow{}\infty} P(A_i)
\]
\textit{Hint:} Use axiom 3.

Let us define the following: $A := \bigcup_{i=1}^{\infty} A_{i}$. We would like to write $A$ in terms of disjoint sets to use axiom 3.
\begin{align}\label{eq:sets:disjoint-sets}
A_{i-1} \subseteq A_i \quad \forall i > 0 \implies A = \bigcup_{i=1}^{\infty} A_{i}\backslash A_{i-1}
\end{align}
where the expression holds if we have $A_{0} = \varnothing$. We regard \ref{eq:sets:disjoint-sets} as starting with $A_1$ and adding the new information from $A_2, A_3,\dots$ (e.g $A_2 \backslash A_1, A_3 \backslash A_2, \dots$). 
\begin{align}\label{eq:sets:axiom3-on-set}
P(A) = P\left(\bigcup_{i=1}^{\infty} A_{i}\backslash A_{i-1}\right) = \sum_{i=1}^\infty P(A_{i}\backslash A_{i-1})&&\text{(by axiom 3)}\\
P(A) = \sum_{i=1}^\infty P(A_{i}\backslash A_{i-1}) = \lim_{n\xrightarrow{}\infty} \sum_{i=1}^n P(A_{i}\backslash A_{i-1}) &&\text{(the infinite summation is a limit)}
\end{align}
From (f), we have $P(A_i) = P(A_{i}\backslash A_{i-1}) + P(A_{i-1}) \implies  P(A_{i}\backslash A_{i-1}) = P(A_i) - P(A_{i-1})$. Then,
\begin{align}
P(A) = \lim_{n\xrightarrow{}\infty} \sum_{i=1}^n P(A_i) - P(A_{i-1}) = \lim_{n\xrightarrow{}\infty} \bigg(\sum_{i=1}^n P(A_i) - \sum_{i=1}^{n-1} P(A_{i})\bigg) = \lim_{n\xrightarrow{}\infty} P(A_n)
\end{align}
where we used $P(A_0) = P(\varnothing) = 0$ from (d). 

In summary:
\begin{align}
P\left(\bigcup_{i=1}^{\infty} A_{i}\right) = P(A) = \lim_{i\xrightarrow{}\infty} P(A_i)
\end{align}

\end{enumerate}

\paragraph{\questionref{q:rv}}
\begin{enumerate}[label=\alph*.]
\item
We choose to represent the outcomes of two dice as integer tuples:
\begin{align*}
\Omega = \{
&(\dA,\dA), (\dA,\dB), (\dA,\dC), (\dA,\dD), (\dA,\dE), (\dA,\dF),  \\
&(\dB,\dA), (\dB,\dB), (\dB,\dC), (\dB,\dD), (\dB,\dE), (\dB,\dF),  \\
&(\dC,\dA), (\dC,\dB), (\dC,\dC), (\dC,\dD), (\dC,\dE), (\dC,\dF),  \\
&(\dD,\dA), (\dD,\dB), (\dD,\dC), (\dD,\dD), (\dD,\dE), (\dD,\dF),  \\
&(\dE,\dA), (\dE,\dB), (\dE,\dC), (\dE,\dD), (\dE,\dE), (\dE,\dF),  \\
&(\dF,\dA), (\dF,\dB), (\dF,\dC), (\dF,\dD), (\dF,\dE), (\dF,\dF) \}
\end{align*}
\item
We define random variables A and B to be:
\begin{align*}
A(s) =
\begin{cases}
1 &\text{, if } s \in \{ (\dA, \dA), (\dA,\dB), (\dA,\dC), (\dA,\dD), (\dA,\dE), (\dA,\dF) \} \\
2 &\text{, if } s \in \{ (\dB, \dA), (\dB,\dB), (\dB,\dC), (\dB,\dD), (\dB,\dE), (\dB,\dF) \} \\
3 &\text{, if } s \in \{ (\dC, \dA), (\dC,\dB), (\dC,\dC), (\dC,\dD), (\dC,\dE), (\dC,\dF) \} \\
4 &\text{, if } s \in \{ (\dD, \dA), (\dD,\dB), (\dD,\dC), (\dD,\dD), (\dD,\dE), (\dD,\dF) \} \\
5 &\text{, if } s \in \{ (\dE, \dA), (\dE,\dB), (\dE,\dC), (\dE,\dD), (\dE,\dE), (\dE,\dF) \} \\
5 &\text{, if } s \in \{ (\dF, \dA), (\dF,\dB), (\dF,\dC), (\dF,\dD), (\dF,\dE), (\dF,\dF) \}
\end{cases} \\
B(s) =
\begin{cases}
1 &\text{, if } s \in \{ (\dA,\dA), (\dB,\dA), (\dC,\dA), (\dD,\dA), (\dE,\dA), (\dF,\dA) \} \\
2 &\text{, if } s \in \{ (\dA,\dB), (\dB,\dB), (\dC,\dB), (\dD,\dB), (\dE,\dB), (\dF,\dB) \} \\
3 &\text{, if } s \in \{ (\dA,\dC), (\dB,\dC), (\dC,\dC), (\dD,\dC), (\dE,\dC), (\dF,\dC) \} \\
4 &\text{, if } s \in \{ (\dA,\dD), (\dB,\dD), (\dC,\dD), (\dD,\dD), (\dE,\dD), (\dF,\dD) \} \\
5 &\text{, if } s \in \{ (\dA,\dE), (\dB,\dE), (\dC,\dE), (\dD,\dE), (\dE,\dE), (\dF,\dE) \} \\
5 &\text{, if } s \in \{ (\dA,\dF), (\dB,\dF), (\dC,\dF), (\dD,\dF), (\dE,\dF), (\dF,\dF) \}
\end{cases}
\end{align*}
We can find the PMFs by counting the number of occurrences in $\Omega$. For instance:
\begin{align*}
p_A(3) = \frac{|\{ (\dC, \dA), (\dC,\dB), (\dC,\dC), (\dC,\dD), (\dC,\dE), (\dC,\dF) \}|}{|\Omega|} = \frac{6}{36} = \frac{1}{6}
\end{align*}
Repeating this for all outcomes gives us the full PDFs:
\begin{align*}
\begin{split}
p_A(x) = \begin{cases}
\frac{1}{6} &\text{, if } x = 1\\
\frac{1}{6} &\text{, if } x = 2\\
\frac{1}{6} &\text{, if } x = 3\\
\frac{1}{6} &\text{, if } x = 4\\
\frac{1}{6} &\text{, if } x = 5\\
\frac{1}{6} &\text{, if } x = 6\\
0 &\text{, otherwise } \\
\end{cases}
\end{split}\text{, }
\begin{split}
p_B(x) = \begin{cases}
\frac{1}{6} &\text{, if } x = 1\\
\frac{1}{6} &\text{, if } x = 2\\
\frac{1}{6} &\text{, if } x = 3\\
\frac{1}{6} &\text{, if } x = 4\\
\frac{1}{6} &\text{, if } x = 5\\
\frac{1}{6} &\text{, if } x = 6\\
0 &\text{, otherwise } \\
\end{cases}
\end{split}
\end{align*}
\item
To show independence of $A$ and $B$ we must show that $p(A \cap B) = p(A)p(B)$.
We have that all outcomes have equal probability $\frac{1}{|\Omega|} = \frac{1}{36}$ and therefore:
\begin{align*}
p(A \cap B) = \frac{1}{36} = \frac{1}{6} \cdot \frac{1}{6} = p(A) p(B)
\end{align*}
\item
We can define a random variable $C = A + B$
\begin{align*}
C(s) =
\begin{cases}
2 &\text{, if } s \in \{ (\dA, \dA) \} \\
3 &\text{, if } s \in \{ (\dA, \dB), (\dB,\dA) \} \\
4 &\text{, if } s \in \{ (\dA, \dC), (\dB,\dB), (\dC,\dA) \} \\
5 &\text{, if } s \in \{ (\dA, \dD), (\dB,\dC), (\dC,\dB), (\dD,\dA) \} \\
6 &\text{, if } s \in \{ (\dA, \dE), (\dB,\dD), (\dC,\dC), (\dD,\dB), (\dE,\dA) \} \\
7 &\text{, if } s \in \{ (\dA, \dF), (\dB,\dE), (\dC,\dD), (\dD,\dC), (\dE,\dB), (\dF,\dA) \} \\
8 &\text{, if } s \in \{ (\dB, \dF), (\dC,\dE), (\dD,\dD), (\dE,\dC), (\dF,\dB) \} \\
9 &\text{, if } s \in \{ (\dC, \dF), (\dD,\dE), (\dE,\dD), (\dF,\dC) \} \\
10 &\text{, if } s \in \{ (\dD, \dF), (\dE,\dE), (\dF,\dD) \} \\
11 &\text{, if } s \in \{ (\dE, \dF), (\dF,\dE) \} \\
12 &\text{, if } s \in \{ (\dF, \dF) \} \\
\end{cases}
\end{align*}
Then the PDF $p_C$ becomes:
\begin{align*}
p_C(x) = \begin{cases}
\frac{1}{36} = \frac{1}{36} &\text{, if } x = 2\\
\frac{2}{36} = \frac{1}{18} &\text{, if } x = 3\\
\frac{3}{36} = \frac{1}{12} &\text{, if } x = 4\\
\frac{4}{36} = \frac{1}{9}  &\text{, if } x = 5\\
\frac{5}{36} = \frac{5}{36} &\text{, if } x = 6\\
\frac{6}{36} = \frac{1}{6}  &\text{, if } x = 7\\
\frac{5}{36} = \frac{5}{36} &\text{, if } x = 8\\
\frac{3}{36} = \frac{1}{9}  &\text{, if } x = 9\\
\frac{3}{36} = \frac{1}{12} &\text{, if } x = 10\\
\frac{2}{36} = \frac{1}{18} &\text{, if } x = 11\\
\frac{1}{36} = \frac{1}{36} &\text{, if } x = 12\\
0 &\text{, otherwise } \\
\end{cases}
\end{align*}
which can be rewritten in more compact form:
\begin{align*}
p_C(x) = \begin{cases}
\frac{6 - |x-6|}{36} &\text{, if } x = \{ 2, 3, \ldots, 12\}\\
0 &\text{, otherwise } \\
\end{cases}
\end{align*}
\end{enumerate}

\paragraph{\questionref{q:crv}}

\begin{enumerate}[label=\alph*.]
\item
\begin{align*}
1 = \int_{-\infty}^{\infty} p(x) \mathrm d x = \int_0^1 Cx \mathrm d x = \frac{1}{2} C x^2 \Big|_{0}^1 = \frac{1}{2}C = 1 \\
\implies C = 2
\end{align*}
\item
\begin{align*}
\mathbb{P}(0.3 \leq X \leq 0.75) = \int_{0.3}^{0.75} 2x \mathrm d x = x^2 \Big|_{0.3}^{0.75} = 0.75^2 - 0.3^2 = 0.4725
\end{align*}
\item
\begin{align*}
\mathbb{P}(X \in [0.3, 0.75] \cup [0.8, 0.9])
&= \int_{0.3}^{0.75} 2x \mathrm d x + \int_{0.8}^{0.9} 2x \mathrm d x \\
&= x^2 \Big|_{0.3}^{0.75} + x^2 \Big|_{0.8}^{0.9} = 0.75^2 - 0.3^2 + 0.9^2 - 0.8^2 = 0.6425
\end{align*}
\item
\begin{align*}
\mathbb{E}_X[X] &= \int x p(x) \mathrm d x = \int_0^1 2 x^2 \mathrm d x = \frac{2}{3} x^3 \Big|_0^1 = \frac{2}{3} \\
\mathbb{E}_X[X^2] &= \int x^2 p(x) \mathrm d x = \int_0^1 2 x^3 \mathrm d x = \frac{2}{4} x^4 \Big|_0^1 = \frac{1}{2} \\
To calculate the variance, we can use the fact that:
\begin{align*}
\mathbb{V}_X[X] &= \mathbb{E}_X[(X - E_X[X])^2] \\
&= \mathbb{E}_X[X^2 - 2X \mathbb{E}_X[X]
\end{align*}


 identity $\mathbb{V}_X[X] = \mathbb{E}[X^2} - (\mathbb{E}[X])^2$ (you should be able to prove this identity!).
\mathbb{V}_X[X] &= \mathbb{E}_X[X^2] - (\mathbb{E}_X[X])^2 = \frac{1}{2} - (\frac{2}{3})^2 = \frac{1}{18} \\
\end{align*}
\end{enumerate}

\paragraph{\questionref{q:jdrv}}

\begin{enumerate}[label=\alph*.]
\item
\begin{tabular}{c|cccccc}
$P(C=c \mid A=a)$ & $a=1$ & $a=2$ & $a=3$ & $a=4$ & $a=5$ & $a=6$ \\
\hline
$c = 2$ & $\frac{1}{6}$ & $0$ & $0$ & $0$ & $0$ & $0$ \\
$c = 3$ & $\frac{1}{6}$ & $\frac{1}{6}$ & $0$ & $0$ & $0$ & $0$ \\
$c = 4$ & $\frac{1}{6}$ & $\frac{1}{6}$ & $\frac{1}{6}$ & $0$ & $0$ & $0$ \\
$c = 5$ & $\frac{1}{6}$ & $\frac{1}{6}$ & $\frac{1}{6}$ & $\frac{1}{6}$ & $0$ & $0$ \\
$c = 6$ & $\frac{1}{6}$ & $\frac{1}{6}$ & $\frac{1}{6}$ & $\frac{1}{6}$ & $\frac{1}{6}$ & $0$ \\
$c = 7$ & $\frac{1}{6}$ & $\frac{1}{6}$ & $\frac{1}{6}$ & $\frac{1}{6}$ & $\frac{1}{6}$ & $\frac{1}{6}$ \\
$c = 8$ & $0$ & $\frac{1}{6}$ & $\frac{1}{6}$ & $\frac{1}{6}$ & $\frac{1}{6}$ & $\frac{1}{6}$ \\
$c = 8$ & $0$ & $0$ & $\frac{1}{6}$ & $\frac{1}{6}$ & $\frac{1}{6}$ & $\frac{1}{6}$ \\
$c = 8$ & $0$ & $0$ & $0$ & $\frac{1}{6}$ & $\frac{1}{6}$ & $\frac{1}{6}$ \\
$c = 8$ & $0$ & $0$ & $0$ & $0$ & $\frac{1}{6}$ & $\frac{1}{6}$ \\
$c = 8$ & $0$ & $0$ & $0$ & $0$ & $0$ & $\frac{1}{6}$ \\
\end{tabular}
\item
\begin{tabular}{c|cccccc}
$P(C=c, A=a)$ & $a=1$ & $a=2$ & $a=3$ & $a=4$ & $a=5$ & $a=6$ \\
\hline
$c = 2$  & $\frac{1}{36}$ & $0$ & $0$ & $0$ & $0$ & $0$ \\
$c = 3$  & $\frac{1}{36}$ & $\frac{1}{36}$ & $0$ & $0$ & $0$ & $0$ \\
$c = 4$  & $\frac{1}{36}$ & $\frac{1}{36}$ & $\frac{1}{36}$ & $0$ & $0$ & $0$ \\
$c = 5$  & $\frac{1}{36}$ & $\frac{1}{36}$ & $\frac{1}{36}$ & $\frac{1}{36}$ & $0$ & $0$ \\
$c = 6$  & $\frac{1}{36}$ & $\frac{1}{36}$ & $\frac{1}{36}$ & $\frac{1}{36}$ & $\frac{1}{36}$ & $0$ \\
$c = 7$  & $\frac{1}{36}$ & $\frac{1}{36}$ & $\frac{1}{36}$ & $\frac{1}{36}$ & $\frac{1}{36}$ & $\frac{1}{36}$ \\
$c = 8$  & $0$ & $\frac{1}{36}$ & $\frac{1}{36}$ & $\frac{1}{36}$ & $\frac{1}{36}$ & $\frac{1}{36}$ \\
$c = 9$  & $0$ & $0$ & $\frac{1}{36}$ & $\frac{1}{36}$ & $\frac{1}{36}$ & $\frac{1}{36}$ \\
$c = 10$ & $0$ & $0$ & $0$ & $\frac{1}{36}$ & $\frac{1}{36}$ & $\frac{1}{36}$ \\
$c = 11$ & $0$ & $0$ & $0$ & $0$ & $\frac{1}{36}$ & $\frac{1}{36}$ \\
$c = 12$ & $0$ & $0$ & $0$ & $0$ & $0$ & $\frac{1}{36}$ \\
\end{tabular}
\item
\begin{align*}
\mathbb{P}(2 \leq C \leq 4) &= 6 \cdot \frac{1}{36} = \frac{1}{6} \\
\mathbb{P}(2 \leq C \leq 4, 2 \leq A \leq 4) &= 3 \cdot \frac{1}{36} = \frac{1}{12} \\
\end{align*}
\end{enumerate}

\paragraph{\questionref{q:mi}}

\begin{enumerate}[label=\alph*.]
\item
\begin{align*}
p(x, y) &= C(x^2 + xy) \\
1 &= \int_{-\infty}^{\infty} \int_{-\infty}^{\infty} C (x^2 + xy) \mathrm d x \mathrm d y = C \int_0^1 \int_0^1 (x^2 + xy) \mathrm d x \mathrm d y \\
&= C \int_0^1 \left( \left( \frac{1}{3} x^3\Big|_{x=0}^1 \right) + \left( \frac{1}{2} y x^2 \Big|_{x=0}^1 \right) \right) \mathrm d y \\
&= C \int_0^1 \left(\frac{1}{3} +\frac{1}{2} y \right) \mathrm d y \\
&= C \left(\frac{1}{3}y \Big|_0^1\right) + C \left(\frac{1}{4} y^2 \Big|_0^1 \right) \\
&= C \left( \frac{1}{3} + \frac{1}{4} \right) = C \frac{7}{12} = 1\\
&\implies C = \frac{12}{7}
\end{align*}
\item
\begin{align*}
\mathbb{P}(0.3 \leq X \leq 0.5) &= \int_0^1 \int_{0.3}^{0.5} \left(Cx^2 + Cxy\right) \mathrm d x \mathrm d y \\
&= C \int_0^1 \left(\left(\frac{1}{3}x^3 \Big|_{0.3}^{0.5} \right) + \left( \frac{1}{2}x^2 y \Big|_{x=0.3}^{0.5} \right) \right) \mathrm d y \\
&= C \int_0^1 \left(\frac{1}{3}(0.5)^3 + \frac{1}{2}(0.5)^2 y - \frac{1}{3}(0.3)^3 - \frac{1}{2}(0.3)^2 y \right) \mathrm d y \\
&= C \int_0^1 \left(\frac{49}{1500} + \frac{2}{25} y \right) \mathrm d y \\
&= C \left( \frac{49}{1500}y \Big|_0^1 \right) + C \left( \frac{2}{50} y^2 \Big|_0^1 \right) \\
&= \frac{12}{27}\left(\frac{49}{1500} + \frac{2}{50} \right) = \frac{109}{875}
\end{align*}
\item

\begin{align*}
\mathbb{P}(X < Y) &= \int_0^1 \int_0^y (Cx^2 + Cxy) \mathrm d x \mathrm d y \\
&= C \int_0^1 \left( \left( \frac{1}{3}x^3 \Big|_0^y \right) + \left( \frac{1}{2} y x^2 \Big|_{x=0}^y \right) \right) \mathrm d y \\
&= C \int_0^1 \left( \frac{1}{3} (y)^3 + \frac{1}{2} y^3 \right) \mathrm d y \\
&= C \left(\frac{1}{3}\cdot\frac{1}{4} y^4 \Big|_0^1 \right) + C \left(\frac{1}{2}\cdot\frac{1}{4} y^4 \Big|_0^1 \right) \\
&= \frac{12}{7} \frac{1}{12} + \frac{12}{7} \frac{1}{8} = \frac{5}{14}
\end{align*}
\begin{align*}
\mathbb{P}(X < Y) &= \int_0^1 \int_x^1 \left( C x^2 + C xy \right) \mathrm dy \mathrm dx \\
&= C \int_0^1 \left( \left( yx^2 \Big|_{y=x}^1 \right) + \left(\frac{1}{2} xy^2 \Big|_{y=x}^1 \right) \right) \mathrm d y \\
&= C \int_0^1 \left( x^2 + \frac{1}{2} x - x^3 - \frac{1}{2} x^3 \right) \mathrm d x \\
&= C \left( ( \frac{1}{3}x^3 + \frac{1}{4}x^2 - \frac{1}{4}x^4 - \frac{1}{8}x^4) \Big|_0^1 \right) \\
&= C \left( \frac{1}{3} + \frac{1}{4} - \frac{1}{4} - \frac{1}{8} \right) = \frac{12}{7} \cdot \frac{5}{24} = \frac{5}{14} 
\end{align*}
\item

\begin{align*}
1 &= \int_0^1 \int_0^1 \int_0^1 C \left(x^2 + xyz \right) \mathrm d x \mathrm d y \mathrm d z \\
&= C \int_0^1 \int_0^1 \left( (\frac{1}{3}x^3 + \frac{1}{2}x^2 yz) \Big|_0^1 \right) \mathrm d y \mathrm d z 
= C \int_0^1 \int_0^1 \left( \frac{1}{3} + \frac{1}{2} yz \right) \mathrm d y \mathrm d z \\
&= C \int_0^1 \left( (\frac{1}{3}y + \frac{1}{4} y^2 z \Big|_0^1 \right) \mathrm d z
= C \int_0^1 \left( \frac{1}{3} + \frac{1}{4} z \right) \mathrm d z \\
&= C \left( (\frac{1}{3} z + \frac{1}{8} z^2 ) \Big|_0^1 \right)
= C \left( \frac{1}{3} + \frac{1}{8} \right) = \frac{11}{24} C = 1 \\
&\implies C = \frac{24}{11}
\end{align*}
\end{enumerate}


\paragraph{\questionref{q:stats-term}}
\begin{enumerate}[label=\alph*.]
\item A statistic is a function that is computed from data. For example, take a data set $X = \{x_1, x_2, x_3, \dots\}$ where we compute the empirical mean $\bar X = \frac{1}{|X|}\sum_n x_n$.
\item An estimator is a function of data that tries to estimate an unknown quantity. Estimators are statistics. Some statistics are also estimators. For example, if we have some data set from that is sampled from some unknown density $p(x)$, then its mean is unknown, and $\bar X$ is an estimator of it.
\item A consistent estimator finds the correct value of the unknown quantity if the dataset grows to infinity. We will prove that $\bar X$ is a consistent estimate of $\int p(x) x \calcd x$ later on in the course.
\item A sample from a random variable is an outcome of the random experiment it represents. For example, you can have a random variable representing the outcome of a coin toss. A sample from it would be heads or tails. We sampled a random variable independently many times, then the outcomes would occur with the frequency specified by the probability distribution of the random variable. Thinking about sampling outcomes from a random variable is often a helpful conceptual technique to think about randomness.
\end{enumerate}

\subsection{Linear Algebra}

\paragraph{Question \ref{q:dot_product}}
$\x^\top \y = 1 \times 0 + (-2) \times 4 + 5 \times (-3) + (-1) \times 7 = 0 + (-8) + (-15) + (-7) = -30$.

\paragraph{Question \ref{q:matrix_product}}
$\y = (24, -14, -12)^\top$, $|| \x ||_2 = \sqrt{23}$, $|| \y ||_2 = \sqrt{916}$.

Note that by definition the $\ell_2$ norm of a vector is $|| \x ||_2 = \sqrt{\x^\top \x}$.

\paragraph{Question \ref{q:basis}} 1, 2. 

A set of vectors $\{\mathbf{b}_1, ..., \mathbf{b}_K \}$ with $\mathbf{b}_k \in \mathbb{R}^d$ can form a basis of $\mathbb{R}^d$ iff $K = d$ the vectors are linearly independent to each other.

\paragraph{Question \ref{q:span}} 2, 5. 

A point $\x \in \mathbb{R}^d$ is in $span(\{\mathbf{b}_1, ..., \mathbf{b}_K \})$ with $\mathbf{b}_k \in \mathbb{R}^d$ iff we can find $a_1, ..., a_K \in \mathbb{R}$ such that $\x = \sum_{k=1}^K a_k \mathbf{b}_k$.

\paragraph{Question \ref{q:rotation_matrix}} The rotation matrix is 
\begin{equation*}
    \begin{pmatrix}
    \cos{\frac{\pi}{4}} & -\sin{\frac{\pi}{4}} \\
    \sin{\frac{\pi}{4}} & \cos{\frac{\pi}{4}}
    \end{pmatrix}.
\end{equation*}

\paragraph{Question \ref{q:linear_equations}}

a) The matrix $A$ and vector $\mathbf{b}$ are
\begin{equation*}
A = \begin{pmatrix}
1 & 2 & 0 \\
3 & 2 & 4 \\
-2 & 1 & -2
\end{pmatrix}, \quad \mathbf{b} = (2, 5, 1)^\top.
\end{equation*}

b) The inverse of $A$ is 
\begin{equation*}
A^{-1} = \begin{pmatrix}
2/3 & -1/3 & -2/3 \\
1/6 & 1/6 & 1/3 \\
7/12 & 5/12 & 1/3
\end{pmatrix}.
\end{equation*}
Therefore we have $\x = A^{-1} \mathbf{b} = (-1, 3/2, 43/12)^\top$.

c) $\text{rank}(A) = 3$: as $A$ is invertible, it must have full rank.

\paragraph{Question \ref{q:eigen_decomp}}

a) When $A$ is symmetric, then $A = Q \Lambda Q^\top$, and $\x^\top A \x = \x^\top Q \Lambda Q^\top \x = (Q^\top \x)^\top \Lambda (Q^\top \x)$. As $Q$ is an orthonormal matrix, we have $\x \rightarrow Q^\top \x$ a one-to-one mapping. Therefore we have
$$\x^\top A \x = \bm{z}^\top \Lambda \bm{z} = \sum_{i=1}^d \lambda_i z_i^2, \quad \bm{z} = (z_1, ..., z_d)^\top = Q^\top \x.$$
Therefore $\x^\top A \x \geq 0 \Leftrightarrow \sum_{i=1}^d \lambda_i z_i^2 \geq 0$. This is true for any $\x \in \mathbb{R}^{d \times 1}$ if and only if $\lambda_i \geq 0$ for all $i = 1,..., d$.

b) We use the permuation invariance property of matrix trace to show the result:
$$Tr(A) = Tr(Q \Lambda Q^{-1}) = Tr(Q^{-1} Q \Lambda) = Tr(\Lambda) = \sum_{i=1}^d \lambda_i.$$

c) We use the product rule of matrix determinant to show the result:
$$det(A) = det(Q \Lambda Q^{-1}) = det(Q) det(\Lambda) det(Q^{-1}) = det(Q) det(\Lambda) det(Q)^{-1} = det(\Lambda) = \prod_{i=1}^d \lambda_i.$$

d) Let us assume the statement is false, i.e., there exists a solution $\lambda^* \neq \lambda_i, \forall i = 1, ..., d$ for the equation $A \bm{q} = \lambda \bm{q}, \bm{q} \neq 0$. Then we can rewrite the equation as
$$A \bm{q} = \lambda^* \bm{q} \quad \Rightarrow \quad (A - \lambda^* I) \bm{q} = \bm{0} \quad \Rightarrow \quad Q (\Lambda - \lambda^* I) Q^{-1} \bm{q} = 0.$$
By definition, the column vectors of $Q$ forms a basis of $\mathbb{R}^d$. Notice that the diagonal entries of $\Lambda - \lambda^* I$ are non-zero as we assume $\lambda^* \neq \lambda_i$. This indicates a contradiction to the assumption of $\bm{q} \neq 0$:
$$Q (\Lambda - \lambda^* I) Q^{-1} \bm{q} = 0 \quad \Rightarrow \quad Q^{-1}\bm{q} = \bm{0} \quad \Rightarrow \quad \bm{q} = \bm{0}.$$






\printbibliography

\end{document}