\begin{question}[Linear transform of a Gaussian random variable] 
\label{q:linear_transform_gaussian}

If $X$ is a $d$-dimensional multivariate Gaussian random variable with mean $\bm{\mu}$ and covariance matrix $\Sigma$, then what is the distribution of the random variable $Y = \BA X$ with an invertible matrix $\BA \in \mathbb{R}^{d \times d}$? 
\end{question}

\begin{question}[Sum of independent Gaussian random variables]
\label{q:sum_of_gaussians}

If $X, Y$ are two independent univariate Gaussian random variables (i.e., $X \ci Y$), show that $Z = X+Y$ is also a univariate Gaussian random variable.
\end{question}

\begin{question}[KL divergence and change-of-variables rule]
\label{q:kl_change_of_variable}

Show that KL divergence is invariant to change-of-variables, i.e., $\mathrm{KL}[p_X(x) || q_X(x)] = \mathrm{KL}[p_Y(y) || q_Y(y)]$ for $Y = T(X)$ with an invertible transformation $T$.
\end{question}

\begin{question}[Independence of Gaussian variables]
\label{q:independence_of_gaussians}

Consider $X = (X_1, ..., X_N)$ as a multivariate random variable, which is distributed as a multivariate Gaussian with covariance matrix $\Sigma$. Show that $X_i \ci X_j | X_{-ij}$ where $X_{-ij}$ collect all the other $X_n$ variables, if for the precision matrix $\Lambda := \Sigma^{-1}$ we have $\Lambda_{ij} = \Lambda_{ji} = 0$.
\end{question}

\begin{question}[Independent vs uncorrelated variables]
\label{q:uncorrelated}

Show that for the following definitions of $X, Y$, these two variables are uncorrelated (i.e., $\mathbb{E}[XY] = \mathbb{E}[X]\mathbb{E}[Y]$), but not independent to each other: $X$ is a univariate Gaussian variable with mean 0, and $Y = X^2$. 

\end{question}


\begin{question}[MML 6.11: Iterated Expectations]
\label{q:mml-6.11}

Consider random variables $X,Y$ with joint distribution $p(x,y)$. Show that
\begin{equation}
\mathbb{E}_X[X]= \mathbb{E}_Y[\mathbb{E}_X [X|Y]]
\end{equation}
where $\mathbb{E}_X [X|Y]$ denotes the expectation under the conditional distribution $p(x|y)$.
\end{question}

\begin{question}[MML 6.13: Probability Integral Transformation]
\label{q:mml-6.13}

Given a continous r.v. $X$, with CDF $F_X(x)$, show that the r.v. $Y:=F_X(X)$ is uniformly distributed.

\end{question}

%\begin{question}[MML 6.12: Manipulation of Gaussian random variables]
%\label{q:mml-6.12}
%
%Consider a Gaussian r.v. $\textbf{x}\sim \mathcal{N}(\textbf{x}|\bm{\mu}_x, \bm{\Sigma}_x)$, where $\textbf{x}\in\mathbb{R}^D$. Consider the following mapping to $\textbf{y}\in\mathbb{R}^E$,
%\begin{equation}
%\textbf{y} = \textbf{A}\textbf{x} + \textbf{b} + \textbf{w}
%\end{equation}
%where $\textbf{A}\in\mathbb{R}^{E\times D}$, $\textbf{b}\in\mathbb{R}^E$, and $\textbf{w}\sim \mathcal{N}(\textbf{w}|\textbf{0}, Q)$. For simplicity, we will assume $\textbf{w}$ is independent, i.e., $Q$ is diagonal.
%\begin{enumerate}
%\item Compute the likelihood $p(\textbf{y}|\textbf{x})$.
%\item Compute the marginal $p(\textbf{y})$.
%\item Given a measurement $\hat{\textbf{y}}$, compute the posterior $p(\textbf{x}|\hat{\textbf{y}})$.
%\end{enumerate}
%
%\end{question}
