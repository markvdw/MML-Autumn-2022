\section{Lecture 13: PCA}

\newcommand{\BB}{\mathbf B}
\newcommand{\BX}{\mathbf X}
\newcommand{\BC}{\mathbf C}
\newcommand{\BU}{\mathbf U}

\begin{question}[Connections to linear auto-encoders]
\label{q:pca_vs_linear_autoencoder}
Consider a linear auto-encoder defined as follows for an input $\x \in \mathbb{R}^{D \times 1}$:
$$\text{Encoder:} \quad \z = enc(\x) = \BB \x, \ \BB \in \mathbb{R}^{M \times D}, M < D, \quad \text{Decoder:} \quad \hat{\x} = dec(\z) = \BA \z, \ \BA \in \mathbb{R}^{D \times M}.$$
Let us assume $rank(\BA) = rank(\BB) = M$.
Given a dataset $\mathcal{D} = \{ \x_n \}_{n=1}^N$ such that $\text{mean}(\x_n) = \bm{0}$ and the covariance matrix computed on $\mathcal{D}$ is invertible, we wish to train the model by minimising the $\ell_2$ reconstruction error:
$$\min_{\BA, \BB} L(\BA, \BB), \quad L(\BA, \BB) := \frac{1}{N} \sum_{n=1}^N || \x_n - \BA \BB \x_n ||_2^2.$$
%
\begin{itemize}
\item[a.] What is the derivative of the objective w.r.t.~$\BA$ and $\BB$?
\item[b.] Given a fixed $\BA$, what is the minimiser solution of $\BB$?
\item[c.] Assume the covariance matrix computed on $\mathcal{D}$ can be eigendecomposed as $\BQ \Lambda \BQ^\top$ with $\Lambda = diag(\lambda_1, ..., \lambda_D), \lambda_1 \geq ... \geq \lambda_D > 0$. Show that $\BA^* = \BQ_{1:M}, \BB^* = \BQ_{1:M}^{\top}$ is a fixed point of the objective $L(\BA, \BB)$, where $\BQ_{1:M}$ contains the first $M$ columns of $\BQ$.
\end{itemize}
%
You might find the following formula useful:
$$\frac{\partial}{\partial \BX} \tr(\BA \BX \BB) = \BA^\top \BB^\top, \quad \frac{\partial}{\partial \BX} \tr(\BX^\top \BB \BX \BC) = \BB \BX \BC + \BB^\top \BX \BC^\top, \quad \frac{\partial}{\partial \BX} \tr(\BX^\top \BX \BB) = \BX \BB + \BX \BB^\top.$$

\end{question}


\begin{question}[SVD and PCA]
\label{q:svd_and_pca}
Assume we are given a dataset $\mathcal{D} = \{ \x_n \}_{n=1}^N$ such that $\text{mean}(\x_n) = \bm{0}$. Write $\BX = [\x_1, ..., \x_n]^\top \in \mathbb{R}^{N \times D}$, demonstrate how to use singular value decomposition (SVD) of $\X$ to obtain PCA solutions of $M$ principal components.  
\end{question}